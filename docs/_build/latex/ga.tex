%% Generated by Sphinx.
\def\sphinxdocclass{report}
\documentclass[letterpaper,10pt,english]{sphinxmanual}
\ifdefined\pdfpxdimen
   \let\sphinxpxdimen\pdfpxdimen\else\newdimen\sphinxpxdimen
\fi \sphinxpxdimen=.75bp\relax

\PassOptionsToPackage{warn}{textcomp}
\usepackage[utf8]{inputenc}
\ifdefined\DeclareUnicodeCharacter
% support both utf8 and utf8x syntaxes
  \ifdefined\DeclareUnicodeCharacterAsOptional
    \def\sphinxDUC#1{\DeclareUnicodeCharacter{"#1}}
  \else
    \let\sphinxDUC\DeclareUnicodeCharacter
  \fi
  \sphinxDUC{00A0}{\nobreakspace}
  \sphinxDUC{2500}{\sphinxunichar{2500}}
  \sphinxDUC{2502}{\sphinxunichar{2502}}
  \sphinxDUC{2514}{\sphinxunichar{2514}}
  \sphinxDUC{251C}{\sphinxunichar{251C}}
  \sphinxDUC{2572}{\textbackslash}
\fi
\usepackage{cmap}
\usepackage[T1]{fontenc}
\usepackage{amsmath,amssymb,amstext}
\usepackage{babel}



\usepackage{times}
\expandafter\ifx\csname T@LGR\endcsname\relax
\else
% LGR was declared as font encoding
  \substitutefont{LGR}{\rmdefault}{cmr}
  \substitutefont{LGR}{\sfdefault}{cmss}
  \substitutefont{LGR}{\ttdefault}{cmtt}
\fi
\expandafter\ifx\csname T@X2\endcsname\relax
  \expandafter\ifx\csname T@T2A\endcsname\relax
  \else
  % T2A was declared as font encoding
    \substitutefont{T2A}{\rmdefault}{cmr}
    \substitutefont{T2A}{\sfdefault}{cmss}
    \substitutefont{T2A}{\ttdefault}{cmtt}
  \fi
\else
% X2 was declared as font encoding
  \substitutefont{X2}{\rmdefault}{cmr}
  \substitutefont{X2}{\sfdefault}{cmss}
  \substitutefont{X2}{\ttdefault}{cmtt}
\fi


\usepackage[Bjarne]{fncychap}
\usepackage{sphinx}

\fvset{fontsize=\small}
\usepackage{geometry}


% Include hyperref last.
\usepackage{hyperref}
% Fix anchor placement for figures with captions.
\usepackage{hypcap}% it must be loaded after hyperref.
% Set up styles of URL: it should be placed after hyperref.
\urlstyle{same}

\addto\captionsenglish{\renewcommand{\contentsname}{Contents:}}

\usepackage{sphinxmessages}
\setcounter{tocdepth}{1}



\title{ga}
\date{Oct 28, 2020}
\release{1.0.0}
\author{Jhonat Heberson}
\newcommand{\sphinxlogo}{\vbox{}}
\renewcommand{\releasename}{Release}
\makeindex
\begin{document}

\pagestyle{empty}
\sphinxmaketitle
\pagestyle{plain}
\sphinxtableofcontents
\pagestyle{normal}
\phantomsection\label{\detokenize{index::doc}}

\index{module@\spxentry{module}!ga@\spxentry{ga}}\index{ga@\spxentry{ga}!module@\spxentry{module}}\index{Genetic (class in ga)@\spxentry{Genetic}\spxextra{class in ga}}

\begin{fulllineitems}
\phantomsection\label{\detokenize{index:ga.Genetic}}\pysiglinewithargsret{\sphinxbfcode{\sphinxupquote{class }}\sphinxcode{\sphinxupquote{ga.}}\sphinxbfcode{\sphinxupquote{Genetic}}}{\emph{\DUrole{n}{goal}\DUrole{o}{=}\DUrole{default_value}{1.0}}, \emph{\DUrole{n}{bounds}\DUrole{o}{=}\DUrole{default_value}{None}}, \emph{\DUrole{n}{mutation\_probability}\DUrole{o}{=}\DUrole{default_value}{0.5}}, \emph{\DUrole{n}{selection\_probability}\DUrole{o}{=}\DUrole{default_value}{0.5}}, \emph{\DUrole{n}{sigma}\DUrole{o}{=}\DUrole{default_value}{0.5}}, \emph{\DUrole{n}{num\_parents}\DUrole{o}{=}\DUrole{default_value}{2}}, \emph{\DUrole{n}{num\_elitism}\DUrole{o}{=}\DUrole{default_value}{2}}, \emph{\DUrole{n}{maxiter}\DUrole{o}{=}\DUrole{default_value}{None}}, \emph{\DUrole{n}{selection\_method}\DUrole{o}{=}\DUrole{default_value}{\textquotesingle{}elitism\textquotesingle{}}}, \emph{\DUrole{n}{cross\_over}\DUrole{o}{=}\DUrole{default_value}{\textquotesingle{}uniform\textquotesingle{}}}, \emph{\DUrole{n}{mutation}\DUrole{o}{=}\DUrole{default_value}{\textquotesingle{}gaussian\textquotesingle{}}}, \emph{\DUrole{n}{submit\_to\_cluster}\DUrole{o}{=}\DUrole{default_value}{False}}}{}~\begin{description}
\item[{Genetic(maxiter=1000, goal=0, cross\_over=’one\_point’,}] \leavevmode
mutation\_probability=0.01, mutation=’uniform’,
selection\_method=’elitism’,num\_parents=2,
num\_elitism=10, bounds=np.array((np.ones(2) * 10 * \sphinxhyphen{}1, np.ones(2) * 10)))

\end{description}

É a classe que é responsavel por realizar a criar a população, 
selecionar os parentes da população, realizar os cruzamentos e mutação.
\begin{description}
\item[{goal}] \leavevmode{[}float, opcional{]}
Define o valor a qual queremos nos aproximar

\item[{bounds}] \leavevmode{[}numpy.ndarray{]}
Define até onde pode ir os valores dos individuo, como inferio e superior

\item[{mutation\_probability}] \leavevmode{[}float{]}
Define a probabilidade de ocorrer a mutação

\item[{selection\_probability}] \leavevmode{[}float{]}
Define a probabilidade de ocorrer a seleção de pais

\item[{sigma}] \leavevmode{[}float{]}
Define a probabilidade do individuo ter mutação

\item[{num\_parents}] \leavevmode{[}integer{]}
Define o numero de parentes que será esolhido no metodo da seleção

\item[{num\_elitism}] \leavevmode{[}integer{]}
Define o numero de pais que será preservado entre as gerações

\item[{maxiter}] \leavevmode{[}integer,opcional{]}
Define o numero de interações maximo que o algoritmo irá fazer para encontrar o resultado

\item[{selection\_method}] \leavevmode{[}str, opcional{]}
Define o metodo de seleção que será usado para escolher os pais da proxima geração

\item[{cross\_over}] \leavevmode{[}str, opcional{]}
Define o metodo de cruzamento que será usado

\item[{mutation}] \leavevmode{[}str, opcional{]}
Define o metodo de mutação que será usado

\item[{submit\_to\_cluster}] \leavevmode{[}bool, opcional{]}
Define se a meta\sphinxhyphen{}hurística será executada no cluster

\end{description}

\begin{sphinxVerbatim}[commandchars=\\\{\}]
\PYG{g+gp}{\PYGZgt{}\PYGZgt{}\PYGZgt{} }\PYG{k+kn}{from} \PYG{n+nn}{fffit} \PYG{k+kn}{import} \PYG{n}{ga}
\PYG{g+gp}{\PYGZgt{}\PYGZgt{}\PYGZgt{} }\PYG{n}{bounds} \PYG{o}{=} \PYG{n}{np}\PYG{o}{.}\PYG{n}{array}\PYG{p}{(}\PYG{p}{(}\PYG{n}{np}\PYG{o}{.}\PYG{n}{ones}\PYG{p}{(}\PYG{l+m+mi}{2}\PYG{p}{)} \PYG{o}{*} \PYG{l+m+mi}{10} \PYG{o}{*} \PYG{o}{\PYGZhy{}}\PYG{l+m+mi}{1}\PYG{p}{,} \PYG{n}{np}\PYG{o}{.}\PYG{n}{ones}\PYG{p}{(}\PYG{l+m+mi}{2}\PYG{p}{)} \PYG{o}{*} \PYG{l+m+mi}{10}\PYG{p}{)}\PYG{p}{)}
\PYG{g+gp}{\PYGZgt{}\PYGZgt{}\PYGZgt{} }\PYG{n}{ga}\PYG{o}{.}\PYG{n}{Genetic}\PYG{p}{(}\PYG{n}{maxiter}\PYG{o}{=}\PYG{l+m+mi}{1000}\PYG{p}{,} \PYG{n}{goal}\PYG{o}{=}\PYG{l+m+mi}{0}\PYG{p}{,} \PYG{n}{cross\PYGZus{}over}\PYG{o}{=}\PYG{l+s+s1}{\PYGZsq{}}\PYG{l+s+s1}{one\PYGZus{}point}\PYG{l+s+s1}{\PYGZsq{}}\PYG{p}{,}
\PYG{g+go}{                   mutation\PYGZus{}probability=0.01, mutation=\PYGZsq{}uniform\PYGZsq{},}
\PYG{g+go}{                   selection\PYGZus{}method=\PYGZsq{}elitism\PYGZsq{},num\PYGZus{}parents=2,}
\PYG{g+go}{                   num\PYGZus{}elitism=10, bounds=bounds)}
\end{sphinxVerbatim}

\begin{sphinxVerbatim}[commandchars=\\\{\}]
\PYG{g+gp}{\PYGZgt{}\PYGZgt{}\PYGZgt{} }
\end{sphinxVerbatim}
\index{calculate\_avg\_fitness() (ga.Genetic method)@\spxentry{calculate\_avg\_fitness()}\spxextra{ga.Genetic method}}

\begin{fulllineitems}
\phantomsection\label{\detokenize{index:ga.Genetic.calculate_avg_fitness}}\pysiglinewithargsret{\sphinxbfcode{\sphinxupquote{calculate\_avg\_fitness}}}{}{}
Calcula a aptidão de media entre todas as populações e salva na lista da classe genética.
\begin{description}
\item[{Returns:}] \leavevmode\begin{quote}\begin{description}
\item[{return}] \leavevmode
void

\end{description}\end{quote}

\end{description}

\end{fulllineitems}

\index{calculate\_best\_fitness() (ga.Genetic method)@\spxentry{calculate\_best\_fitness()}\spxextra{ga.Genetic method}}

\begin{fulllineitems}
\phantomsection\label{\detokenize{index:ga.Genetic.calculate_best_fitness}}\pysiglinewithargsret{\sphinxbfcode{\sphinxupquote{calculate\_best\_fitness}}}{}{}
Calcula a melhor aptidão entre todas as populações e salva na lista da classe genética.
\begin{description}
\item[{Returns:}] \leavevmode\begin{quote}\begin{description}
\item[{return}] \leavevmode
void

\end{description}\end{quote}

\end{description}

\end{fulllineitems}

\index{calculate\_pop\_fitness() (ga.Genetic method)@\spxentry{calculate\_pop\_fitness()}\spxextra{ga.Genetic method}}

\begin{fulllineitems}
\phantomsection\label{\detokenize{index:ga.Genetic.calculate_pop_fitness}}\pysiglinewithargsret{\sphinxbfcode{\sphinxupquote{calculate\_pop\_fitness}}}{\emph{\DUrole{n}{func}}}{}
calcula a aptidão da população e retorna em lista da classe genética.
\begin{description}
\item[{Returns:}] \leavevmode\begin{quote}\begin{description}
\item[{return}] \leavevmode
void

\end{description}\end{quote}

\end{description}

\end{fulllineitems}

\index{cross\_over\_one\_point() (ga.Genetic method)@\spxentry{cross\_over\_one\_point()}\spxextra{ga.Genetic method}}

\begin{fulllineitems}
\phantomsection\label{\detokenize{index:ga.Genetic.cross_over_one_point}}\pysiglinewithargsret{\sphinxbfcode{\sphinxupquote{cross\_over\_one\_point}}}{\emph{\DUrole{n}{population}}}{}
Esta função realiza o cruzamento de um ponto no gene.
\begin{description}
\item[{Args:}] \leavevmode
:param population:(\sphinxcode{\sphinxupquote{list}}): Lists one containing a population.
:param parents :(\sphinxcode{\sphinxupquote{float}}): parents gene to carry out the mutation.

\end{description}

Returns:

\end{fulllineitems}

\index{cross\_over\_two\_points() (ga.Genetic method)@\spxentry{cross\_over\_two\_points()}\spxextra{ga.Genetic method}}

\begin{fulllineitems}
\phantomsection\label{\detokenize{index:ga.Genetic.cross_over_two_points}}\pysiglinewithargsret{\sphinxbfcode{\sphinxupquote{cross\_over\_two\_points}}}{\emph{\DUrole{n}{population}}}{}
Esta função realiza o cruzamento de dois pontos no gene.
\begin{description}
\item[{Args:}] \leavevmode
:param population:(\sphinxcode{\sphinxupquote{list}}): Lists one containing a population.
:param parents :(\sphinxcode{\sphinxupquote{float}}): parents gene to carry out the mutation.

\end{description}

Returns:

\end{fulllineitems}

\index{cross\_over\_uniform() (ga.Genetic method)@\spxentry{cross\_over\_uniform()}\spxextra{ga.Genetic method}}

\begin{fulllineitems}
\phantomsection\label{\detokenize{index:ga.Genetic.cross_over_uniform}}\pysiglinewithargsret{\sphinxbfcode{\sphinxupquote{cross\_over\_uniform}}}{\emph{\DUrole{n}{population}}}{}
Esta função realiza o cruzamento uniforme no gene.
\begin{description}
\item[{Args:}] \leavevmode
:param population:(\sphinxcode{\sphinxupquote{list}}): Lists one containing a population.
:param parents :(\sphinxcode{\sphinxupquote{float}}): parents gene to carry out the mutation.

\end{description}

Returns:

\end{fulllineitems}

\index{do\_full\_step() (ga.Genetic method)@\spxentry{do\_full\_step()}\spxextra{ga.Genetic method}}

\begin{fulllineitems}
\phantomsection\label{\detokenize{index:ga.Genetic.do_full_step}}\pysiglinewithargsret{\sphinxbfcode{\sphinxupquote{do\_full\_step}}}{\emph{\DUrole{n}{func}}, \emph{\DUrole{o}{**}\DUrole{n}{kwargs}}}{}
Execute uma etapa completa de GA.

Este método passa por todos os outros métodos para realizar uma completa
Etapa GA, para que possa ser chamada a partir de um loop no método run ()..

\end{fulllineitems}

\index{elitism\_selection() (ga.Genetic method)@\spxentry{elitism\_selection()}\spxextra{ga.Genetic method}}

\begin{fulllineitems}
\phantomsection\label{\detokenize{index:ga.Genetic.elitism_selection}}\pysiglinewithargsret{\sphinxbfcode{\sphinxupquote{elitism\_selection}}}{}{}
Função que realiza seleção elitista.
\begin{description}
\item[{Args:}] \leavevmode
:param population:(\sphinxcode{\sphinxupquote{list}}): Lists one containing a population.

\item[{Returns:}] \leavevmode
:return:population(\sphinxcode{\sphinxupquote{list}}): An element of the population list select.

\end{description}

\end{fulllineitems}

\index{evaluate\_single\_fitness\_test() (ga.Genetic method)@\spxentry{evaluate\_single\_fitness\_test()}\spxextra{ga.Genetic method}}

\begin{fulllineitems}
\phantomsection\label{\detokenize{index:ga.Genetic.evaluate_single_fitness_test}}\pysiglinewithargsret{\sphinxbfcode{\sphinxupquote{evaluate\_single\_fitness\_test}}}{\emph{\DUrole{n}{func}}, \emph{\DUrole{n}{enum\_particles}\DUrole{o}{=}\DUrole{default_value}{False}}, \emph{\DUrole{n}{add\_step\_num}\DUrole{o}{=}\DUrole{default_value}{False}}, \emph{\DUrole{o}{**}\DUrole{n}{kwargs}}}{}
Execute a função fornecida como o teste de aptidão para todas as partículas.
\begin{description}
\item[{fun}] \leavevmode{[}callable{]}
The fitness test function to be minimized:
\begin{quote}

\sphinxcode{\sphinxupquote{func(individual.ichromosome, **kwargs) \sphinxhyphen{}\textgreater{} float}}.
\end{quote}

\item[{enum\_particles}] \leavevmode{[}boolean{]}
If \sphinxtitleref{True}, the population will be enumerated and the individual index will
be passed to \sphinxtitleref{func} as keyword \sphinxtitleref{part\_idx}, added to \sphinxtitleref{kwargs}

\item[{add\_step\_num}] \leavevmode{[}boolean{]}
If \sphinxtitleref{True}, the current step number will be passed to \sphinxtitleref{func}
as keyword \sphinxtitleref{step\_num}, added to \sphinxtitleref{kwargs}

\end{description}

{\color{red}\bfseries{}**}kwargs: Other keywords to the fitness function, will be passed as is.

\end{fulllineitems}

\index{fitness\_variation() (ga.Genetic method)@\spxentry{fitness\_variation()}\spxextra{ga.Genetic method}}

\begin{fulllineitems}
\phantomsection\label{\detokenize{index:ga.Genetic.fitness_variation}}\pysiglinewithargsret{\sphinxbfcode{\sphinxupquote{fitness\_variation}}}{\emph{\DUrole{n}{fitness\_evaluation}}}{}
Essa função realiza a verificação da variação do fitness entre as populações
\begin{description}
\item[{Args:}] \leavevmode
fitness\_evaluation ({[}type{]}): {[}description{]}

\end{description}

\end{fulllineitems}

\index{mutation\_binary() (ga.Genetic method)@\spxentry{mutation\_binary()}\spxextra{ga.Genetic method}}

\begin{fulllineitems}
\phantomsection\label{\detokenize{index:ga.Genetic.mutation_binary}}\pysiglinewithargsret{\sphinxbfcode{\sphinxupquote{mutation\_binary}}}{\emph{\DUrole{n}{population}}}{}
A função realiza a mutação binária e retorna a população com a modificação,
vale ressaltar que esta mutação só é válida para população binária.
\begin{quote}
\begin{description}
\item[{Args:}] \leavevmode
:param population:(\sphinxcode{\sphinxupquote{list}}): Lists one containing a population.

\end{description}
\end{quote}

Returns:

\end{fulllineitems}

\index{mutation\_gaussian() (ga.Genetic method)@\spxentry{mutation\_gaussian()}\spxextra{ga.Genetic method}}

\begin{fulllineitems}
\phantomsection\label{\detokenize{index:ga.Genetic.mutation_gaussian}}\pysiglinewithargsret{\sphinxbfcode{\sphinxupquote{mutation\_gaussian}}}{\emph{\DUrole{n}{population}}}{}
A função realiza a mutação de Gausiana e retorna a população com a modificada.
\begin{description}
\item[{Args:}] \leavevmode
:param population:(\sphinxcode{\sphinxupquote{list}}): Lists one containing a population.

\end{description}

Returns:

\end{fulllineitems}

\index{mutation\_uniform() (ga.Genetic method)@\spxentry{mutation\_uniform()}\spxextra{ga.Genetic method}}

\begin{fulllineitems}
\phantomsection\label{\detokenize{index:ga.Genetic.mutation_uniform}}\pysiglinewithargsret{\sphinxbfcode{\sphinxupquote{mutation\_uniform}}}{\emph{\DUrole{n}{population}}}{}
A função realiza a mutação uniforme e retorna a população modificada.
\begin{quote}
\begin{description}
\item[{Args:}] \leavevmode
:param population:(\sphinxcode{\sphinxupquote{list}}): Lists one containing a population.

\end{description}
\end{quote}

Returns:

\end{fulllineitems}

\index{populate() (ga.Genetic method)@\spxentry{populate()}\spxextra{ga.Genetic method}}

\begin{fulllineitems}
\phantomsection\label{\detokenize{index:ga.Genetic.populate}}\pysiglinewithargsret{\sphinxbfcode{\sphinxupquote{populate}}}{\emph{\DUrole{n}{size\_population}}, \emph{\DUrole{n}{bounds}\DUrole{o}{=}\DUrole{default_value}{None}}, \emph{\DUrole{n}{x0}\DUrole{o}{=}\DUrole{default_value}{None}}, \emph{\DUrole{n}{ndim}\DUrole{o}{=}\DUrole{default_value}{None}}, \emph{\DUrole{n}{sigma}\DUrole{o}{=}\DUrole{default_value}{None}}, \emph{\DUrole{n}{type\_create}\DUrole{o}{=}\DUrole{default_value}{\textquotesingle{}uniform\textquotesingle{}}}}{}
Retorna uma lista consistindo de vários indivíduos que formam a população.
\begin{description}
\item[{Return:}] \leavevmode\begin{quote}\begin{description}
\item[{return}] \leavevmode
void

\end{description}\end{quote}

\end{description}

\end{fulllineitems}

\index{random\_selection() (ga.Genetic method)@\spxentry{random\_selection()}\spxextra{ga.Genetic method}}

\begin{fulllineitems}
\phantomsection\label{\detokenize{index:ga.Genetic.random_selection}}\pysiglinewithargsret{\sphinxbfcode{\sphinxupquote{random\_selection}}}{}{}
Esta função realiza uma seleção do torneio e retorno gene vencedor.
\begin{description}
\item[{Args:}] \leavevmode
:param population:(\sphinxcode{\sphinxupquote{list}}): Lists one containing a population.

\item[{Returns:}] \leavevmode\begin{quote}\begin{description}
\item[{return}] \leavevmode
sub\_population(\sphinxcode{\sphinxupquote{list}}): An element of the sub\_population list select.

\end{description}\end{quote}

\end{description}

\end{fulllineitems}

\index{roulette\_selection() (ga.Genetic method)@\spxentry{roulette\_selection()}\spxextra{ga.Genetic method}}

\begin{fulllineitems}
\phantomsection\label{\detokenize{index:ga.Genetic.roulette_selection}}\pysiglinewithargsret{\sphinxbfcode{\sphinxupquote{roulette\_selection}}}{}{}
Esta função realiza a seleção por releta do gene a ser realizado a mutação.
\begin{description}
\item[{Args:}] \leavevmode\begin{quote}\begin{description}
\item[{param population}] \leavevmode
population:(\sphinxcode{\sphinxupquote{list}}): Lists one containing a population.

\end{description}\end{quote}

\item[{Returns:}] \leavevmode\begin{quote}\begin{description}
\item[{return}] \leavevmode
selected(\sphinxcode{\sphinxupquote{list}}): An element of the population list select.

\end{description}\end{quote}

\end{description}

\end{fulllineitems}

\index{run() (ga.Genetic method)@\spxentry{run()}\spxextra{ga.Genetic method}}

\begin{fulllineitems}
\phantomsection\label{\detokenize{index:ga.Genetic.run}}\pysiglinewithargsret{\sphinxbfcode{\sphinxupquote{run}}}{\emph{\DUrole{n}{func}}, \emph{\DUrole{n}{DEBUG}\DUrole{o}{=}\DUrole{default_value}{None}}, \emph{\DUrole{o}{**}\DUrole{n}{kwargs}}}{}
Execute uma execução de otimização completa.

Faz a otimização com a execução da atualização das velocidades
e as coordenadas também verifica o critério interrompido para encontrar fitnnes.
\begin{description}
\item[{func}] \leavevmode{[}callable{]}
Function that calculates fitnnes.

\end{description}
\begin{quote}

The dictionary that stores the optimization results.
\end{quote}

\end{fulllineitems}

\index{selection() (ga.Genetic method)@\spxentry{selection()}\spxextra{ga.Genetic method}}

\begin{fulllineitems}
\phantomsection\label{\detokenize{index:ga.Genetic.selection}}\pysiglinewithargsret{\sphinxbfcode{\sphinxupquote{selection}}}{}{}
{[}summary{]}
\begin{description}
\item[{Raises:}] \leavevmode
ValueError: {[}description{]}

\item[{Returns:}] \leavevmode
{[}type{]}: {[}description{]}

\end{description}

\end{fulllineitems}

\index{sorted\_population() (ga.Genetic static method)@\spxentry{sorted\_population()}\spxextra{ga.Genetic static method}}

\begin{fulllineitems}
\phantomsection\label{\detokenize{index:ga.Genetic.sorted_population}}\pysiglinewithargsret{\sphinxbfcode{\sphinxupquote{static }}\sphinxbfcode{\sphinxupquote{sorted\_population}}}{\emph{\DUrole{n}{population}}}{}
Selecione o gene a ser realizado a mutação.
\begin{description}
\item[{Args:}] \leavevmode
:param population:(\sphinxcode{\sphinxupquote{list}}): Lists one containing a population.

\item[{Returns:}] \leavevmode\begin{quote}\begin{description}
\item[{return}] \leavevmode
score(\sphinxcode{\sphinxupquote{list}}): An element of the population list select.

\end{description}\end{quote}

\end{description}

\end{fulllineitems}

\index{update\_swarm() (ga.Genetic method)@\spxentry{update\_swarm()}\spxextra{ga.Genetic method}}

\begin{fulllineitems}
\phantomsection\label{\detokenize{index:ga.Genetic.update_swarm}}\pysiglinewithargsret{\sphinxbfcode{\sphinxupquote{update\_swarm}}}{}{}
Atualize a população realizando cruzamento, mutação
\begin{description}
\item[{Returns:}] \leavevmode
:return population(\sphinxcode{\sphinxupquote{list}} of \sphinxcode{\sphinxupquote{Particles}}): returns a list of
swarms.

\end{description}

\end{fulllineitems}


\end{fulllineitems}

\index{Individual (class in ga)@\spxentry{Individual}\spxextra{class in ga}}

\begin{fulllineitems}
\phantomsection\label{\detokenize{index:ga.Individual}}\pysiglinewithargsret{\sphinxbfcode{\sphinxupquote{class }}\sphinxcode{\sphinxupquote{ga.}}\sphinxbfcode{\sphinxupquote{Individual}}}{\emph{\DUrole{n}{x0}\DUrole{o}{=}\DUrole{default_value}{np.zeros(2)}}, \emph{\DUrole{n}{ndim}\DUrole{o}{=}\DUrole{default_value}{2}}, \emph{\DUrole{n}{bounds}\DUrole{o}{=}\DUrole{default_value}{np.array(np.ones(2) * 10 * \sphinxhyphen{} 1, np.ones(2) * 10)}}, \emph{\DUrole{n}{type\_create}\DUrole{o}{=}\DUrole{default_value}{\textquotesingle{}uniform\textquotesingle{}}}}{}
Cria os individuos que compoem a população da classe Genetic
\begin{description}
\item[{fitness}] \leavevmode{[}None or flout{]}
é o melhor valor do individuo

\item[{size\_individual}] \leavevmode{[}int {]}
é o tamanho do indiviuo

\item[{create\_individual}] \leavevmode{[}class ‘fffit.ga.Individual’{]}
cria o individuo com suas caracteristicas

\end{description}
\begin{description}
\item[{type\_create}] \leavevmode{[}str, optional{]}
Define qual será o tipo de criação da população inicial

\item[{x0}] \leavevmode{[}np.ndarray, optional{]}
Define qual o ponto inicial até os limites de onde vai criar o valor do indviduo

\item[{bounds}] \leavevmode{[}numpy.ndarray{]}
Define até onde pode ir os valores dos individuo, como inferio e superior

\item[{ndim}] \leavevmode{[}integer{]}
Define quantos dimensões tem o indiviuo ou seja o tamanho do array

\item[{sigma :float, opcional}] \leavevmode
Define a probabilidade do individuo ter mutação

\end{description}

\begin{sphinxVerbatim}[commandchars=\\\{\}]
\PYG{g+gp}{\PYGZgt{}\PYGZgt{}\PYGZgt{} }\PYG{k+kn}{from} \PYG{n+nn}{fffit} \PYG{k+kn}{import} \PYG{n}{ga}
\PYG{g+gp}{\PYGZgt{}\PYGZgt{}\PYGZgt{} }\PYG{k+kn}{import} \PYG{n+nn}{numpy} \PYG{k}{as} \PYG{n+nn}{np}
\PYG{g+gp}{\PYGZgt{}\PYGZgt{}\PYGZgt{} }\PYG{n}{ranges} \PYG{o}{=} \PYG{l+m+mi}{2}
\PYG{g+gp}{\PYGZgt{}\PYGZgt{}\PYGZgt{} }\PYG{n}{ndim} \PYG{o}{=} \PYG{l+m+mi}{2}
\PYG{g+gp}{\PYGZgt{}\PYGZgt{}\PYGZgt{} }\PYG{n}{bounds} \PYG{o}{=} \PYG{n}{np}\PYG{o}{.}\PYG{n}{array}\PYG{p}{(}\PYG{p}{(}\PYG{n}{np}\PYG{o}{.}\PYG{n}{ones}\PYG{p}{(}\PYG{l+m+mi}{2}\PYG{p}{)} \PYG{o}{*} \PYG{l+m+mi}{10} \PYG{o}{*} \PYG{o}{\PYGZhy{}}\PYG{l+m+mi}{1}\PYG{p}{,} \PYG{n}{np}\PYG{o}{.}\PYG{n}{ones}\PYG{p}{(}\PYG{l+m+mi}{2}\PYG{p}{)} \PYG{o}{*} \PYG{l+m+mi}{10}\PYG{p}{)}\PYG{p}{)}
\PYG{g+gp}{\PYGZgt{}\PYGZgt{}\PYGZgt{} }\PYG{n}{individual} \PYG{o}{=} \PYG{n}{ga}\PYG{o}{.}\PYG{n}{Individual}\PYG{p}{(}\PYG{n}{x0}\PYG{o}{=}\PYG{n}{np}\PYG{o}{.}\PYG{n}{zeros}\PYG{p}{(}\PYG{l+m+mi}{2}\PYG{p}{)}\PYG{p}{,} \PYG{n}{ndim}\PYG{o}{=}\PYG{l+m+mi}{2}\PYG{p}{,} \PYG{n}{bounds}\PYG{o}{=}\PYG{n}{bounds}\PYG{p}{,} \PYG{n}{sigma}\PYG{o}{=}\PYG{k+kc}{None}\PYG{p}{,} \PYG{n}{type\PYGZus{}create}\PYG{o}{=}\PYG{l+s+s1}{\PYGZsq{}}\PYG{l+s+s1}{uniform}\PYG{l+s+s1}{\PYGZsq{}}\PYG{p}{)}
\PYG{g+go}{\PYGZlt{}fffit.ga.Individual object at 0x7f8968797c18\PYGZgt{}}
\PYG{g+gp}{\PYGZgt{}\PYGZgt{}\PYGZgt{} }\PYG{n}{individual}\PYG{o}{.}\PYG{n}{chromosome}
\PYG{g+go}{array([ 5.7287427 , \PYGZhy{}0.54066483])}
\end{sphinxVerbatim}
\index{create\_individual() (ga.Individual method)@\spxentry{create\_individual()}\spxextra{ga.Individual method}}

\begin{fulllineitems}
\phantomsection\label{\detokenize{index:ga.Individual.create_individual}}\pysiglinewithargsret{\sphinxbfcode{\sphinxupquote{create\_individual}}}{\emph{type\_create=uniform}, \emph{x0=np.zeros(2)}, \emph{np.array((np.ones(2) * 10 * \sphinxhyphen{}1}, \emph{np.ones(2) * 10))}, \emph{sigma=None}}{}
Cria os chromosome que pertence a classe Individual
\begin{description}
\item[{type\_create}] \leavevmode{[}str, optional{]}
Define qual será o tipo de criação da população inicial

\item[{x0}] \leavevmode{[}np.ndarray, optional{]}
Define qual o ponto inicial até os limites de onde vai criar o valor do indviduo

\item[{bounds}] \leavevmode{[}numpy.ndarray{]}
Define até onde pode ir os valores dos individuo, como inferio e superior

\item[{sigma :float, opcional}] \leavevmode
Define a probabilidade do individuo ter mutação

\end{description}

\end{fulllineitems}


\end{fulllineitems}



\chapter{Indices and tables}
\label{\detokenize{index:indices-and-tables}}\begin{itemize}
\item {} 
\DUrole{xref,std,std-ref}{genindex}

\item {} 
\DUrole{xref,std,std-ref}{modindex}

\item {} 
\DUrole{xref,std,std-ref}{search}

\end{itemize}


\renewcommand{\indexname}{Python Module Index}
\begin{sphinxtheindex}
\let\bigletter\sphinxstyleindexlettergroup
\bigletter{g}
\item\relax\sphinxstyleindexentry{ga}\sphinxstyleindexpageref{index:\detokenize{module-ga}}
\end{sphinxtheindex}

\renewcommand{\indexname}{Index}
\printindex
\end{document}